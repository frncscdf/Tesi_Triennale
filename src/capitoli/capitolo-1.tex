% !TEX encoding = UTF-8
% !TEX TS-program = pdflatex
% !TEX root = ../tesi.tex

%**************************************************************
\chapter{L'azienda}
\label{cap:azienda}
%**************************************************************

%**************************************************************
\section{Profilo aziendale}
    \textbf{Zextras s.r.l.} nasce a Torri di Quartesolo(VI) nel 2011 come estensione di \textbf{Studio Storti s.r.l.} che opera, dal 1997, nel campo delle soluzioni \gls{opensg}. Sin dall'inizio, l'obiettivo principale di questa società era quello di estendere \gls{zcsg} \textit{Open Source Edition}, uno dei più diffusi strumenti collaborativi per aziende e pubbliche amministrazioni, aggiungendo nuove funzionalità. \\
    Nel corso degli anni è nata e cresciuta \textbf{Zextras Suite}, una raccolta di estensioni che permettono di arricchire \gls{zcsg} con nuove funzionalità utili nel suo utilizzo in ambito professionale.
    Le soluzioni proposte da \textbf{Zextras} con i sui prodotti vengono da subito apprezzate da \textbf{Synacor}, l'azienda che sviluppa e mantiene \gls{zcsg}, la quale decide di includere parte del codice sviluppato dall'azienda nella sua versione \gls{opensg}. Attualmente i prodotti sviluppati dall'azienda vengono utilizzati da più di 100 milioni di utenti in tutto il mondo.

    \begin{figure}[h]
        \centering
        \includegraphics[width=0.55\textwidth]{immagini/zextras_logo.png}
        \caption{Logo Zextras}
        \label{fig: Logo Zextras}
    \end{figure}

\section{Dominio applicativo}
    \subsection{Zimbra Open Source Edition}
        \textbf{Zextras}, come già accennato, è nata con l'obiettivo di creare nuovi contenuti per \gls{zcsg} facendone quindi il suo \textit{core business}.
        \textbf{Zimbra} è un software collaborativo di gruppo adatto a coordinare e supportare l'attività lavorativa sia di aziende sia di pubbliche amministrazioni. I principali servizi offerti da questo software sono i seguenti:
        \begin{itemize}
            \item posta elettronica;
            \item gestione calendari condivisi e organizzazione eventi;
            \item interfaccia amministratore;
            \item supporto dei servizi su dispositivi mobili.
        \end{itemize}
        Per estendere l'applicativo con ulteriori funzionalità sviluppate anche da terze parti, è possibile installare un \gls{pluging} che in ambiente \gls{zcsg} viene chiamato \textbf{Zimlet}. \\
        Esistono due versioni di \gls{zcsg}:
        \begin{itemize}
            \setlength\itemsep{0em}
            \item \textbf{Zimbra Open Source Edition}: è la versione su cui lavora \textbf{Zextras} e offre i servizi elecanti in precedenza;
            \item \textbf{Zimbra Network Edition}: è una versione a pagamento che offre alcune funzionalità \gls{closedg} tra cui un protocollo per la sincronizzazione di calendario e contatti e maggiori funzionalità per gli amministratori.
        \end{itemize}
    \subsection{Zextras Suite}
        \textbf{Zextras Suite} è un'insieme di funzionalità che permettono di aggiungere delle funzionalità a \gls{zcsg} \textit{Open Source Edition} in modo indipendente da quest'ultimo. Ciò permette una configurazione altamente modulare e personalizzabile in base alle necessità dell'utilizzatore. \\
        Questa suite offre i seguenti prodotti:
        \begin{itemize}
            \setlength\itemsep{0em}
            \item \textbf{Powerstore}: sistema di ottimizzazione dei dati che permette il risparmio di memoria sui server \textbf{Zimbra};
            \item \textbf{Backup}: motore di \gls{backupg} in \gls{realtimeg};
            \item \textbf{Admin}: strumenti dedicati agli amministratori per la gestione e il monitoraggio dei servizi attivi sull'istanza di \gls{zcsg};
            \item \textbf{Mobile}: gestione e sincronizzazione di posta elettronica, contatti, eventi e calendario su dispositivi mobili tramite protocolli \textit{Exchange} e \textit{EAS 16.0 (ActiveSync)};
            \item \textbf{Chat}: piattaforma di messaggistica istantanea nativamente integrata in \gls{zcsg}, che permette lo scambio di messaggi e di effettuare videochiamate;
            \item \textbf{Drive}: piattaforma per la condivisione di file e l'utilizzo di fogli di lavoro in modo condiviso.
        \end{itemize}
\section{Struttura interna}
L'azienda è suddivisa in diversi settori specializzati, ciò permette di avere un organico fornito di tutte le competenze necessarie per raggiungere gli obiettivi aziendali. Di seguito un elenco che illustra i diversi settori:
\begin{itemize}
    \item \textbf{Commercio}: ...;
    \item \textbf{Amministrazione di sistema}: ...
    \item \textbf{Team di sviluppo}:
        \begin{itemize}
            \item \textbf{Front-end}: ...;
            \item \textbf{Back-end}: ...;
            \item \textbf{UI/UX Design}: ...;
            \item \textbf{Mobile}: ....
        \end{itemize}
\end{itemize}

\section{Processi aziendali}
\subsection{Fornitura}
\subsection{Comunicazione}
\subsubsection{Strumenti utilizzati}

\subsection{Metodologia di sviluppo}

\subsection{Gestione di progetto}
\subsubsection{Strumenti utilizzati}

\subsection{Configurazione}
Controllo di versione
\subsubsection{Strumenti utilizzati}
Bitbucket e Git

\subsection{Documentazione}
\paragraph{Strumenti utilizzati}

\subsection{Verifica}
\subsubsection{Strumenti utilizzati}
Jenkins, ecc...


Descrizione della metodologia adottata in particolare nel contesto aziendale

%**************************************************************