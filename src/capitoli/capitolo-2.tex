% !TEX encoding = UTF-8
% !TEX TS-program = pdflatex
% !TEX root = ../tesi.tex

%**************************************************************
\chapter{Obiettivi dello stage}
\label{cap:obiettivi}
%**************************************************************

% SSO IMAGE SOURCE https://www.getkisi.com/courses/sso-guide

\section{Presentazione del progetto}
    Questo progetto nasce dall'esigenza di avere un sistema di autenticazione per la \textit{webmail} di \gls{zcsg} che fornisse il giusto connubio tra sicurezza e facilità d'uso. \textbf{Zextras} utilizza molti strumenti e servizi al suo interno, sia per lo sviluppo sia per l'amministrazione. Tutti questi servizi utilizzano \gls{okta} per la gestione dell'autenticazione, ciò significa che è sufficiente effettuare una sola autenticazione con il proprio account su \gls{okta} per poter poi accedere a tutti i servizi ad esso collegati. \\
    Questa tipologia di autenticazione si chiama \gls{sso} e consiste nell'utilizzo di una singola credenziale che permette di accedere a più servizi diversi. \\
    \gls{zcsg} era l'unico serivizio, uno dei più estensivamente utilizzati in azienda, che non beneficiava di questa tecnica di autenticazione. Per questo motivo \textbf{Zextras} prende la decisione di esplorare la possibilità di svilupparne una personalizzata che, oltre all'autenticazione base, avesse le seguenti caratteristiche:
    \begin{itemize}
        %\setlength\itemsep{0em}
        \item quando un utente che possiede un \textit{account} su \gls{okta} ma non su \gls{zcsg}, accede per la prima volta su quest'ultima, viene creato un nuovo account su \gls{zcsg}, associato a quello di \gls{okta};
        \item importazione su \gls{zcsg} delle informazioni utente presenti su \gls{okta};
        \item sincronizzazione dei gruppi di \gls{okta} ai quali un utente appartiene, con liste di distribuzione e classi di servizio di \gls{zcsg}.
    \end{itemize}
    Il fine di questo progetto era quello di uniformare il metodo di autenticazione di \gls{zcsg} rispetto a tutti gli servizi utilizzati dall'azienda e soprattutto avere un sistema personalizzato e configurabile che permette di automatizzare alcune operazioni ripetitive e dispendiose in termini di tempo.
    Per poter proporre una soluzione conforme alle esigenze emerse, ho dovuto per prima cosa condurre uno studio e un'analisi dei protocolli di autenticazione presenti sul mercato. La seconda parte invece consisteva nella progettazione e implementazione del sistema utilizzando il protocollo più idoneo.
    
    \begin{figure}[h]
        \centering
        \includegraphics[width=0.75\textwidth]{immagini/sso.png}
        \caption{Single Sign-On}
        \textbf{Fonte}:
        \href{https://developer.fourth.com/en-gb/docs/single-sign-saml}{developer.fourth.com}
        \label{fig: Single Sign-On}
    \end{figure}

\newpage

    \subsection{Analisi stato dell'arte protocolli di autenticazione}\label{sec:att_analisi}
    %Descrizione dell'attività di analisi e di studio dei protocolli di autenticazione che ho dovuto condurre come prima attività durante lo stage.
    La prima attività che ho dovuto svolgere era la ricerca e lo studio dei protocolli di autenticazione più diffusi e utilizzati sul mercato. L'azienda conosceva già alcuni di questi, ma voleva avere un'analisi approfondita e degli scenari pratici. Inoltre poteva rivelarsi una valida occasione per scoprire nuovi possibili protocolli adatti all'implementazione del sistema di autenticazione. \\
    L'azienda ha quindi richiesto di redigere un documento che riportasse tutti i risultati delle mie ricerche, in particolare l'analisi dei pro e dei contro di ciascun protocollo e se questo potesse essere idoneo al nostro caso d'uso.

    \subsection{Progettazione di un sistema di autenticazione personalizzato}
        In seguito all'analisi svolta, il secondo obiettivo dello stage era la progettazione, seguita dall'implementazione, di un sistema di autenticazione per \gls{zcsg} utilizzando il protocollo ritenuto più idoneo. Il sistema di autenticazione doveva essere in grado di:
        \begin{itemize}
            \setlength\itemsep{0em}
            \item utilizzare \gls{okta} come \gls{idpg};
            \item supportare altri \gls{idpg} che utilizzato lo stesso protocollo;
            \item supportare \gls{zcsg} tramite il \gls{ssog} di \gls{okta} rendendo opzionale l'accesso tramite \textit{email} e \textit{password};
            \item garantire sicurezza.
        \end{itemize}
    \newpage
    \begin{figure}[h]
        %\setlength{\textfloatsep}{1pt}
        %\setlength{\belowcaptionskip}{1pt}
        \centering
        \includegraphics[width=0.75\textwidth]{immagini/rd.jpg}
        \caption{\textit{Research \& Development}}
        \textbf{Fonte}:
        \href{https://youteam.io/blog/research-and-development-the-fourth-pillar-of-software-development/}{youteam.io}
        \label{fig: Research & Development}
    \end{figure}
    
\section{Vantaggi aziendali}
    \textbf{Zextras} ospita stage di questa tipologia da ormai diversi anni nonostante sia un'attività che richiede un quantità di tempo e impegno non indifferente poiché è costretta a sottrarre risorse dai progetti e dalle attività in corso che, molto spesso, sono vincolate da scadenze. Tuttavia i vantaggi portati da uno strumento come lo stage sono molteplici. Per prima cosa l'azienda ha la possibilità di condurre dei progetti di ricerca che molto spesso coinvolgono l'utilizzo di nuove tecnologie, senza essere costretta a rallentare lo sviluppo dei prodotti ordinari e soprattutto riducendo il rischio di investire troppe risorse in progetti che potenzialmente potrebbero non proseguire. \\
    Un altro aspetto interessante dal punto di vista aziendale è il poter entrare in contatto con studenti che, seppur privi di esperienza lavorativa, sono spesso in grado di proporre soluzioni alternative e creative a problemi comuni. I vantaggi elencati si sono effettivamente concretizzati nel tempo, infatti negli utlimi anni il team di sviluppo ha integrato nel suo organico alcuni studenti che, una volta completato lo stage, sono rimasti a contribuire alle sfide tecnologiche che l'azienda affronta giornalmente. \\
    %Alcuni progetti di stage proposti sono molto strategici e hanno altri obiettivi oltre al progetto in sè. Per esempio, l'azienda può sfruttare questo strumento per poter inserire un nuovo 
    Ciò significa che uno stage ben organizzato e seguito si rivela essere un vero e proprio investimento, non solo per l'azienda che lo ospita, la quale beneficerà degli \textit{output} di questo strumento, ma anche per la crescita degli studenti, i quali saranno le figure professionali che nel futuro faranno parte dell'intera industria che al giorno d'oggi necessita di molte risorse vista la sua dinamicità.

\newpage

\section{Vincoli}
    \subsection{Vincoli metodologici}
        Per lo svolgimento dell'attività di stage è stato deciso, in comune accordo con il \textit{tutor}, che il modo più efficace per portarla a termine fosse lavorare presso la sede aziendale. La prima motivazione deriva dalla metodologia di sviluppo adottata, descritta nella sezione \secref{sec:met_svilpuppo}, la quale è fortemente incentrata sulla comunicazione e sul confronto con tutti i membri nel \textit{team}. Inoltre, poiché sarei stato affiancato da alcuni \textit{senior developer} si trattava di un'ottima occasione per apprendere il più possibile da professionisti nel settore. \\
        Oltre agli incontri di allineamento giornalieri, il \textit{Project Manager} ha stabilito che ci sarebbero stati degli incontri formali, insieme al \textit{tutor} aziendale, per fare il punto della situazione. Nella fase conclusiva del progetto era inoltre prevista una \textit{\gls{demog}} a cui avrebbero presenziato:
        \begin{itemize}
            \item il \textit{CEO} dell'azienda;
            \item il \textit{Project Manager};
            \item il responsabile tecnico dell'azienda;
            \item il \textit{team} di sviluppo con il \textit{tutor} aziendale;
            \item il responsabile di un altro \textit{team};
        \end{itemize}
        La presenza di figure appartenti a diversi settori e \textit{team} sarebbe state utile a discutere il prodotto sviluppato, sia dal punto di vista funzionale sia da quello infrastrutturale. Inoltre, ricevere un \textit{feedback} da punti di vista diversi è certamente utile per il miglioramento dell'applicazione nel suo insieme.
        
        \begin{figure}[h]
            \centering
            \includegraphics[width=0.8\textwidth]{immagini/team.jpg}
            \caption{\textit{Teamwork}}
            \textbf{Fonte}:
            \href{https://www.sandler.com/blog/6-benefits-of-teamwork-in-the-workplace/}{sandler.com}
            \label{fig: Teamwork}
        \end{figure}
        
    \subsection{Vincoli temporali}
    L'attività di stage prevista aveva una durata di 304 ore, da svolgere nell'arco di due mesi, suddivise in 8 settimane della durata di circa 40 ore ciascuna. L'orario di lavoro accordato con l'azienda era dal Lunedì al Venerdì dalle ore 9.00 alle 18.00. \\
    Prima dell'inizio dello stage ho pianificato, insieme al \textit{tutor} le attività per ciascuna settimana di lavoro, facendone una stima orario per il completamento. Sin dal primo giorno io e il \textit{tutor} aziendale abbiamo rispettato il piano di lavoro stabilito. Tuttavia, in seguito all'attività di analisi svolta, come descritto nella sezione \secref{sec:att_analisi}, abbiamo dovuto dedicare più tempo per ottenere un prototipo base funzionante, poiché non era ancora chiaro come utilizzare il protocollo di autenticazione scelto. Nonostante un rallentamento a metà percorso, il resto dello stage ha avuto un andamento lineare che mi ha permesso di terminare il lavoro senza fretta. 
    
    \begin{figure}[h]
        \centering
        \includegraphics[width=1\textwidth]{immagini/plan.png}
        \caption{\textit{Planning}}
        \textbf{Fonte}:
        \href{https://www.sysaid.com/blog/entry/8-tips-on-how-to-plan-for-configuration-management-part-1}{sysaid.com}
        \label{fig: Planning}
    \end{figure}
    
    \subsection{Vincoli tecnologici}
    Le tecnologie e i linguaggi utilizzati duarante questo progetto dello \textit{stack} tecnologico dell'azienda e sono le seguenti:
    \begin{itemize}
        \item \textbf{Java}: essendo il linguaggio con cui è scritto \gls{zcsg}, ne consegue che tutta la tecnologia di \textbf{Zextras} si sia adeguata, compreso il mio progetto;
        \item \textbf{Git}: sistema di versionamento, come descritto nella sezione \secref{sec:configurazione};
        \item \textbf{Docker}: è una tecnologia che permette di creare, rilasciare ed eseguire delle applicazioni utilizzando i \gls{containerg}. In questo modo il \gls{containerg} potrà essere eseguito, tramite \textit{docker}, su macchine diverse che solitamente sono configurate in diversi modi. Questo comportamente è simile alla virtualizzazione e permette di rendere gli ambienti di esecuzione solidi e deterministici. In particolare l'azienda lo utilizza per creare delle istanze di \gls{zcsg} utilizzate come ambienti di test in fase di sviluppo. In azienda i \gls{containerg} di \textit{docker} sono gestiti tramite un servizio di nome \textbf{\textit{Portainer}}\footnote{https://www.portainer.io/}.
    \end{itemize}

\section{Aspettative aziendali}
%Descrizione degli output che l'azienda si aspettava da questo stage
Al termine delle 304 ore previste per il completamento dello stage, l'azienda si aspettava di avere almeno un prototipo funzionante che facesse uso di un protocollo di autenticazione, al fine di permettere l'autenticazione su \gls{zcsg} tramite l'\gls{idpg} \gls{okta}. Come già descritto in precedenza però, la parte interessante di questo progetto era l'aggiunta di alcune funzionalità personalizzate ad un sistema di autenticazione standard. Per questo motivo dopo aver concluso l'attività di ricerca sui protocolli, descritta nel paragrafo \secref{sec:att_analisi}, ho discusso, insieme al \textit{team} e al \textit{Project Manager}, i requsiti specifici da implementare. Ciò è stato utile perché al momento della stesura del piano di lavoro, alcuni di essi non potevano essere a me chiari, causa mancanza di contesto riguardante l'ambiente \gls{zcsg}. Gli obiettivi stabiliti erano suddivisi, in ordine di importanza, in \textbf{obbligatori} e \textbf{desiderabili}.
\paragraph{Obiettivi obbligatori}
\begin{itemize}
    \item Analisi stato dell'arte dei protocolli di autenticazione più diffusi;
    \item Implemenazione di un sistema di autenticazione per \gls{zcsg} tramite il protocollo scelto;
\end{itemize}
\paragraph{Obiettivi desiderabili}
\begin{itemize}
    \item \textit{\gls{prov}};
    \item Importazione dei dati dell'utente da \gls{okta} a \gls{zcsg};
    \item Flusso di autenticazione a partire sia dall'\gls{idpg} sia da \gls{zcsg};
    \item Controllo delle classi di servizio di \gls{zcsg} tramite l'\gls{idpg};
    \item Gestione delle liste di distribuzione di \gls{zcsg} tramite l'\gls{idpg};
    \item Autenticazione a due fattori;
    \item Integrazione con \textit{WebAuthn}\footnote{https://www.w3.org/TR/webauthn-2/}
\end{itemize}

\section{Aspettative personali}
Descrizione di ciò che io mi aspettavo al momento della scelta di questo stage

