%**************************************************************
% Glossario
%**************************************************************
%\renewcommand{\glossaryname}{Glossario}

\newglossaryentry{zcsg}
{
    name=\glslink{zcsg}{Zimbra Collaboration},
    text=Zimbra,
    sort=zimbra,
    description={\textit{Software} collaborativo che offre servizi come posta elettronica e calendario condiviso. \'E possibile estenderlo con nuove funzionalità tramite meccanismi chiamati \textit{Zimlet}, simili al concetto di \gls{pluging}}
}

\newglossaryentry{spg}
{
    name=\glslink{sp}{Service Provider},
    text=service provider,
    sort=service provider,
    description={\'E un sistema che fornisce un servizio a degli utenti. Lo si può intendere come un sito \textit{web} che eroga un certo servizio}
}

\newglossaryentry{idpg}
{
    name=\glslink{idp}{Identity Provider},
    text=identity provider,
    sort=identity provider,
    description={Un \textit{identity provider} è un sistema che crea, mantiene e gestisce le informazioni sull'identità di un utente. Si occupa di fornire il servizio di autenticazione ai \gls{spg}}
}

\newglossaryentry{okta}
{
    name=\glslink{okta}{Okta},
    text=Okta,
    sort=okta,
    description={Okta è una società di gestione di identità e di accessi, quindi un \gls{idpg}}
}

\newglossaryentry{xmlg}
{
    name=\glslink{xml}{XML},
    text=XML,
    sort=xml,
    description={Linguaggio di \textit{Markup} che consente la definizione di metadati}
}

\newglossaryentry{samlg}
{
    name=\glslink{saml}{SAML (Secuity Assertion Markup Language)},
    text=SAML,
    sort=saml,
    description={\'E un protocollo bastato su \gls{xmlg} che permette lo scambio di messaggi per effettuare autenticazione e autorizzazione tra domini distinti. Tipicamente gli attori del procollo sono un \gls{idpg} che fornisce l'identità dell'utente da autenticare e un \gls{spg} che fornisce il servizio a cui l'utente ha richiesto l'accesso o una risorsa}
}

\newglossaryentry{ssog}
{
    name=\glslink{sso}{SSO (Single Sign-On)},
    text=Single Sign-On,
    sort=sso,
    description={Si tratta di un sistema di autenticazione che permette ad un utente di effettuare un'unica autenticazione, valida per più servizi e/o risorse che lo supportano. Questo permette all'utente di avere un'unica credenziale valida per più servizi indipendenti}
}

\newglossaryentry{samlass}
{
    name=\glslink{samlass}{SAML Assertion},
    text=SAML Assertion,
    sort=saml assertion,
    description={Una asserzione \gls{samlg} è un documento in formato \gls{xmlg} che contiene le informazioni sull'autenticazione e/o autorizzazione di un utente. Tale documento è solitamente generato dall'\gls{idpg} e inviato al \gls{spg}}
}

\newglossaryentry{crudg}
{
    name=\glslink{crud}{CRUD},
    text=CRUD,
    sort=crud,
    description={Questo acronimo viene spesso usato in ambito di \textit{database management} e indica:
    \begin{itemize}
        \item \textit{\textbf{Create}}: creazione di un utente;
        \item \textit{\textbf{Read}}: richiesta attributi di un utente;
        \item \textit{\textbf{Update}}: aggiornamento attributi di un utente;
        \item \textit{\textbf{Delete}}: non si parla di una cancellazione vera e propria di un utente ma di \textit{deprovisioning}, ovvero una disabilitazione dell'\textit{account} di quest'ultimo o di un cambio di permessi
    \end{itemize}}
}

\newglossaryentry{prov}
{
    name=\glslink{prov}{Provisioning},
    text=Provisioning,
    sort=provisioning,
    description={Con il termine \textit{provisioning} si intende, generalmente, la gestione degli utenti. Questo termine include un insieme di funzionalità riassunte dall'acronimo \gls{crudg}}
}

\newglossaryentry{opensg}
{
    name=\glslink{opensg}{Open Source},
    text=open source,
    sort=open source,
    description={Con il termine \textit{open source} si fa riferimento ad un \textit{software} la cui lincenza permette di utilizzarlo, modificarlo e redistribuirlo}
}

\newglossaryentry{closedg}
{
    name=\glslink{closedg}{Closed Source},
    text=closed source,
    sort=closed source,
    description={Con questo termine si fa riferimento ad un \textit{software} proprietario utilizzabile sotto certe condizioni. Di solito non è possibile modificarlo, condividerlo e ridistribuirlo}
}

\newglossaryentry{pluging}
{
    name=\glslink{pluging}{Plug-in},
    text=plug-in,
    sort=plugin,
    description={Componente \textit{software} che aggiunge funzionalità all'applicazione su cui viene installato}
}

\newglossaryentry{backupg}
{
    name=\glslink{backupg}{Backup},
    text=backup,
    sort=backup,
    description={Quando si parla di \textit{backup}, si fa riferimento al processo di duplicazione di dati su più supporti (fisici o \textit{cloud}) al fine di poterli recuperare in caso di perdita inattesa}
}

\newglossaryentry{realtimeg}
{
    name=\glslink{realtimeg}{Real-time},
    text=real-time,
    sort=realtime,
    description={\textit{Software} che opera sotto condizioni temporali ben definite}
}

\newglossaryentry{userexpg}
{
    name=\glslink{userexpg}{User experience},
    text=user experience,
    sort=user experience,
    description={Il \textit{feedback} manifestato dall'utente nell'interagire con un certo prodotto, sistema o servizio}
}

\newglossaryentry{agileg}
{
    name=\glslink{agileg}{Agile},
    text=agile,
    sort=agile,
    description={Approccio allo sviluppo software che pone il focus sul consegnare al cliente un \textit{software} completo, funzionante  e di qualità in tempi brevi}
}

\newglossaryentry{frameworkg}
{
    name=\glslink{frameworkg}{Framework},
    text=framework,
    sort=framework,
    description={Insieme di strumenti che definiscono la struttura di un sistema a livello concettuale. Nel caso del \textit{software} si può intedendere come un'architettura sulla quale basare lo sviluppo di un prodotto. Quando si parla di modello di sviluppo si intende l'insieme di strumenti teorici che permettono di mettere in atto un concetto specifico}
}

\newglossaryentry{taskg}
{
    name=\glslink{taskg}{Task},
    text=task,
    sort=task,
    description={Incarico di piccole dimensioni assegnato al soggetto che dovrà portarlo a termine}
}

\newglossaryentry{wysiwygg}
{
    name=\glslink{wysiwygg}{WYSIWYG (What You See Is What You Get)},
    text=WYSIWYG,
    sort=wysiwyg,
    description={Si riferisce ad una tipologia di \textit{editor} di testo in grado di mostrare in tempo reale, durante la scrittura, quale sarà l'aspetto finale del documento}
}

\newglossaryentry{repositoryg}
{
    name=\glslink{repositoryg}{Repository},
    text=repository,
    sort=repository,
    description={Nell'ambito dello sviluppo \textit{software} rappresenta un contenitore di codice sorgente, gestito da un sistema di versionamento}
}

\newglossaryentry{workflowg}
{
    name=\glslink{workflowg}{Workflow},
    text=workflow,
    sort=workflowg,
    description={Flusso di esecuzione di un insieme di attività}
}

\newglossaryentry{codereviewg}
{
    name=\glslink{codereviewg}{Code Review},
    text=code review,
    sort=code review,
    description={Attività di revisione del codice effttuata da persone diverse dagli autori, al fine di correggere errori, migliorarne la qualità ed eventualmente proporre soluzioni alternative}
}

\newglossaryentry{itsg}
{
    name=\glslink{itsg}{Issue Tracking System},
    text=issue tracking system,
    sort=issue tracking system,
    description={\textit{Software} che permette di gestire in maniera ordinata un insieme di \textit{issue}, ovvero dei \gls{taskg} da svolgere. Tipicamente viene utilizzato in ambito collaborativo in quanto permette di tenere traccia delle \textit{issue} portate a termine da tutti i membri del \textit{team}}
}

\newglossaryentry{cig}
{
    name=\glslink{cig}{Continuous Integration},
    text=continuous integration,
    sort=continuous integration,
    description={Pratica nell'ambito dell'ingeneria del \textit{software} che consiste nell'integrazione frequente del lavoro svolto negli ambienti locali degli sviluppatori verso l'ambiente condiviso, ovvero il \textit{\gls{repositoryg}} in remoto}
}

\newglossaryentry{cdg}
{
    name=\glslink{cdg}{Continuous Delivery},
    text=continuous delivery,
    sort=continuous delivery,
    description={Pratica nell'ambito dell'ingeneria del \textit{software} che consiste nel rilasciare la \textit{build} di un \textit{software} pronta per l'ambiente di produzione}
}

\newglossaryentry{bugg}
{
    name=\glslink{bugg}{Bug},
    text=bug,
    sort=bug ,
    description={In informatica si tratta di un errore \textit{software} che produce risultati inattesi}
}

\newglossaryentry{cloudg}
{
    name=\glslink{cloudg}{Cloud},
    text=cloud,
    sort=cloud,
    description={In informatica si tratta di un errore nel software che produce risultati inattesi}
}

\newglossaryentry{demog}
{
    name=\glslink{demog}{Demo},
    text=demo,
    sort=demo,
    description={Dimostrazione di in tempo reale di un prodotto, in questo caso di un \textit{software}}
}

\newglossaryentry{containerg}
{
    name=\glslink{containerg}{Container},
    text=container,
    sort=container,
    description={Un \textit{Docker} \textit{container} è un'unità \textit{software} che contiene un ambiente di esecuzione completo di librerie, dipendenze e configurazioni che è in grado di essere eseguito in modo sicuro e deterministico in altri ambienti \textit{Docker} ospitati su diverse macchine}
}

\newglossaryentry{cosg}
{
    name=\glslink{cosg}{Class of Service},
    text=classe di servizio,
    sort=classe di servizio,
    description={In \gls{zcsg}, una classe di servizio è un identificativo per un gruppo di utenti che condividono un insieme di permessi e proprietà}
}

\newglossaryentry{distlistg}
{
    name=\glslink{distlistg}{Distribution List},
    text=liste di distribuzione,
    sort=liste di distribuzione,
    description={In \gls{zcsg}, una lista di ditribuzione è un meccanismo simile alla \textit{mailing list} classica disponibile sui \textit{server} di posta elettronica}
}

\newglossaryentry{restg}
{
    name=\glslink{restg}{REST API},
    text=REST API,
    sort=rest api,
    description={Sono una tipologia di \gls{apig} che utilizzano richieste \gls{httpg} per lo scambio di informazioni}
}

\newglossaryentry{javadocg}
{
    name=\glslink{javadocg}{Javadoc},
    text=javadoc,
    sort=javadoc,
    description={Strumento che permette di generare la documentazione in formato \textit{HTML} per il linguaggio \textit{Java}}
}

\newglossaryentry{openidg}
{
    name=\glslink{openidg}{OpenID},
    text=OpenID,
    sort=openid,
    description={\'E un protocollo di autenticazione che permette di verificare l'identità di un utente tramite l'ausilio di un \gls{idpg} che scambia informazioni con il\gls{spg} nelle stesse modalità di una \gls{restg}. In particolare, le informazioni riguardanti l'autenticazione di un utente, sono gestite tramite un \gls{jsong} \textit{Web Token}\footnote{https://jwt.io/}}
}

\newglossaryentry{xmlschemag}
{
    name=\glslink{xmlschemag}{XML-Schema},
    text=XML-Schema,
    sort=xml schema,
    description={Rappresenta la definizione della struttura di un particolare documento \gls{xmlg}. Tutti i documenti che utilizzano un certo \textit{schema} devono adeguarsi alle regole in esso definito}
}

\newglossaryentry{oauthg}
{
    name=\glslink{oauthg}{OAuth 2.0},
    text=OAuth 2.0,
    sort=oauthg,
    description={Si tratta di un protocollo di autorizzazione, ovvero permette di autorizzare un utente ad accedere ad una particolare risorsa}
}

\newglossaryentry{endpointg}
{
    name=\glslink{endpointg}{Endpoint},
    text=endpoint,
    sort=endpoint,
    description={L'\textit{endpoint} è un \textit{URL} tramite il quale è possibile raggiungere un servizio}
}

\newglossaryentry{jsong}
{
    name=\glslink{jsong}{JSON (JavaScript Object Notation)},
    text=JSON,
    sort=json,
    description={\'E uno \textit{standard} che definisce un formato di \textit{file} che permette di rappresentare oggetti composti da coppie chiave-valore ed \textit{array}. Questo formato è molto utilizzato nello scambio di dati tra \textit{client} e \textit{server} e viene spesso preferito al formato \gls{xmlg}, il quale risulta più verboso}
}

\newglossaryentry{parsingg}
{
    name=\glslink{parsingg}{Parsing},
    text=parsing,
    sort=parsing,
    description={In informatica, è un processo che stabilisce se un testo scritto con i simboli di un determinato linguaggio sia conforme ad esso. Questo processo viene eseguito da un \textit{software} chiamato \textit{parser}}
}

\newglossaryentry{struttdatig}
{
    name=\glslink{struttdatig}{Struttura dati},
    text=struttura dati,
    sort=struttura dati,
    description={In informatica, è un'entita che viene utilizzata per organizzare un insieme di dati in memoria (\textit{RAM} o di massa). Vengono pesantemente utilizzate per la progettazione di algoritmi efficienti}
}

\newglossaryentry{libg}
{
    name=\glslink{libg}{Libreria},
    text=libreria,
    sort=libreria,
    description={In informatica, una libreria è un insieme di funzioni, metodi e strutture dati che possono essere incluse in un altro modulo \textit{software} rispettando opportune precondizioni}
}

\newglossaryentry{httpg}
{
    name=\glslink{httpg}{HTTP (Hypertext Transfer Protocol)},
    text=HTTP,
    sort=http,
    description={Protocollo di applicazione alla base dello scambio di informazioni tra \textit{client} e \textit{server} all'interno dei servizi \textit{web}}
}

\newglossaryentry{soapg}
{
    name=\glslink{soapg}{SOAP (Simple Object Access Protocol)},
    text=SOAP,
    sort=soap,
    description={Protocollo per lo scambio di messaggi tra componenti \textit{software}. L'idea risiede nello strutturare i messaggi secondo il paradigma della programmazione ad oggetti}
}

\newglossaryentry{apig}
{
    name=\glslink{apig}{API (Application Programming Interface)},
    text=API,
    sort=api,
    description={Nell'ambito dello sviluppo \textit{software} si intende un insieme di procedure, opportunamente organizzate, che risolvono un determinato problema. Spesso si usa questo termine per rifersi alle librerie offerte da un linguaggio di programmazione}
}

\newglossaryentry{depinjg}
{
    name=\glslink{depinjg}{Dependency Injection},
    text=dependency injection,
    sort=dependency injection,
    description={In ingegneria del \textit{software} è una tecnica nella quale un oggetto si occupa di fornire tuttle le dipendenze necessarie ad un altro oggetto. Viene utilizzato per semplificare l'attività di \textit{test} sul \textit{software}}
}

%**************************************************************
% Acronimi
%**************************************************************
\renewcommand{\acronymname}{Acronimi e abbreviazioni}

\newacronym[description={\glslink{zcsg}{Zimbra Collaboration Suite}}]
    {zcs}{ZCS}{Zimbra Collaboration Suite}
    
\newacronym[description={\glslink{idpg}{Identity Provider}}]
    {idp}{IdP}{Identity Provider}

\newacronym[description={\glslink{spg}{Identity Provider}}]
    {sp}{SP}{Service Provider}
    
\newacronym[description={\glslink{samlg}{Security Assertion Markup Language}}]
    {saml}{Security Assertion Markup Language}{SAML}
    
\newacronym[description={\glslink{xmlg}{eXtensible Markup Language}}]
    {xml}{XML}{eXtensible Markup Language}
    
\newacronym[description={\glslink{ssog}{Single Sign-On}}]
    {sso}{SSO}{Single Sign-On}
    
\newacronym[description={\glslink{crudg}{Create-Read-Update-Delete}}]
    {crud}{CRUD}{Create-Read-Update-Delete}