%**************************************************************
% Glossario
%**************************************************************
%\renewcommand{\glossaryname}{Glossario}

\newglossaryentry{zcsg}
{
    name=\glslink{zcs}{ZCS},
    text=Zimbra,
    sort=zimbra,
    description={Zimbra Collaboration}
}

\newglossaryentry{spg}
{
    name=\glslink{sp}{SP},
    text=SP,
    sort=service provider,
    description={Un service provider è un sistema che fornisce un servizio a degli utenti. Lo si può identificare come un sito web che eroga un certo servizio}
}

\newglossaryentry{idpg}
{
    name=\glslink{idp}{IdP},
    text=IdP,
    sort=identity provider,
    description={Un identity provider è un sistema che crea, mantiene e gestisce le informazioni sull'identità di un utente. Si occupa di fornire il servizio di autenticazione ai \gls{spg}}
}

\newglossaryentry{okta}
{
    name=\glslink{okta}{Okta},
    text=Okta,
    sort=okta,
    description={Okta è una società di gestione di identità e di accessi, quindi un \gls{idpg}}
}

\newglossaryentry{xmlg}
{
    name=\glslink{xml}{XML},
    text=XML,
    sort=xml,
    description={Linguaggio di \textbf{Markup} che consente la definizione di \textbf{metadati}}
}

\newglossaryentry{samlg}
{
    name=\glslink{saml}{SAML},
    text=SAML,
    sort=saml,
    description={SAML è un protocollo bastato su \gls{xmlg} che permette lo scambio di messaggi per effettuare autenticazione e autorizzazione tra domini distinti. Tipicamente gli attori del procollo sono un \gls{idpg} che fornisce l'identità dell'utente da autenticare e un \gls{spg} che fornisce il servizio a cui l'utente ha richiesto l'accesso. }
}

\newglossaryentry{ssog}
{
    name=\glslink{sso}{SSO},
    text=Single Sign-On,
    sort=sso,
    description={Si tratta di un sistema di autenticazione che permette ad un utente di effettuare un'unica autenticazione, valida per più servizi e/o risorse ai quali è abilitato. Questo permette all'utente di avere un'unica credenziale valida per più servizi indipendenti}
}

\newglossaryentry{samlass}
{
    name=\glslink{samlass}{SAML Assertion},
    text=SAML Assertion,
    sort=saml assertion,
    description={Una asserzione \gls{samlg} è un documento in formato \gls{xmlg} che contiene le informazioni sull'autenticazione e/o autorizzazione di un utente. Tale documento è solitamente generato dall'\gls{idpg} e inviato al \gls{spg}}
}

\newglossaryentry{crudg}
{
    name=\glslink{crud}{CRUD},
    text=CRUD,
    sort=crud,
    description={Questo acronimo viene spesso usato in ambito di \textbf{database management} e indica:
    \begin{itemize}
        \item \textbf{Create}: creazione di un utente;
        \item \textbf{Read}: richiesta attributi di un utente;
        \item \textbf{Update}: aggiornamento attributi di un utente;
        \item \textbf{Delete}: non si parla di una cancellazione vera e propria di un utente ma di \textbf{deprovisioning}, ovvero una disabilitazione dell'account di quest'ultimo o di un cambio di permessi
    \end{itemize}}
}

\newglossaryentry{prov}
{
    name=\glslink{prov}{Provisioning},
    text=Provisioning,
    sort=provisioning,
    description={Con il termine \textbf{provisioning} si intende, generalmente, la gestione degli utenti. Questo termine include un insieme di funzionalità riassunte dall'acronimo \gls{crudg}}
}

\newglossaryentry{opensg}
{
    name=\glslink{opensg}{Open Source},
    text=open source,
    sort=open source,
    description={Con il termine \textbf{open source} si fa riferimento ad un software la cui lincenza permette di utilizzarlo, modificarlo e redistribuirlo}
}

\newglossaryentry{closedg}
{
    name=\glslink{closedg}{Closed Source},
    text=closed source,
    sort=closed source,
    description={Con il termine \textbf{closed source} si fa riferimento ad un software proprietario, quindi non disponibile pubblicamente}
}

\newglossaryentry{pluging}
{
    name=\glslink{pluging}{Plug-in},
    text=plug-in,
    sort=plugin,
    description={Componente software che aggiunge funzionalità all'applicazione su cui viene installato}
}

\newglossaryentry{backupg}
{
    name=\glslink{backupg}{Backup},
    text=backup,
    sort=backup,
    description={Quando si parla di backup, si fa riferimento al processo di duplicazione di dati su più supporti (fisici o cloud) al fine di poterli recuperare in caso di perdita inattesa}
}

\newglossaryentry{realtimeg}
{
    name=\glslink{realtimeg}{Real-time},
    text=real-time,
    sort=realtime,
    description={Software che opera sotto condizioni temporali ben definite}
}

%**************************************************************
% Acronimi
%**************************************************************
\renewcommand{\acronymname}{Acronimi e abbreviazioni}

\newacronym[description={\glslink{zcsg}{Zimbra Collaboration Suite}}]
    {zcs}{ZCS}{Zimbra Collaboration Suite}
    
\newacronym[description={\glslink{idpg}{Identity Provider}}]
    {idp}{IdP}{Identity Provider}

\newacronym[description={\glslink{spg}{Identity Provider}}]
    {sp}{SP}{Service Provider}
    
\newacronym[description={\glslink{samlg}{Security Assertion Markup Language}}]
    {saml}{SAML}{Security Assertion Markup Language}
    
\newacronym[description={\glslink{xmlg}{eXtensible Markup Language}}]
    {xml}{XML}{eXtensible Markup Language}
    
\newacronym[description={\glslink{ssog}{Single Sign-On}}]
    {sso}{SSO}{Single Sign-On}
    
\newacronym[description={\glslink{crudg}{Create-Read-Update-Delete}}]
    {crud}{CRUD}{Create-Read-Update-Delete}